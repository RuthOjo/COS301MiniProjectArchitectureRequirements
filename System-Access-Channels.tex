System Access Channels is all the external services or applications that have to interact with the Buzz system in some way or another to ensure that all functionality is available to the users. The main systems that will be used in the current project scope are:

\begin{enumerate}
	\item The UP Computer Science Website:
	\begin{itemize}
		\item The Computer Science website must be able to integrate with the system allowing students to access it from module pages providing these users with help and support in their studies.
		\item This system will eventually replace the current outdated CS forum that is being used.
	\end{itemize}
	\item LDAP:
	\begin{itemize}
		\item LDAP is the database used by the University to store and record all of the students information. This database can be modified or expanded to hold user ranks and post information.
		\item This means that access and permissions to the database by the Buzz system need to be provided by the database administrators. 
	\end{itemize}
	\item The Internet:
	\begin{itemize}
		\item Since this system will be integrated with the CS website as mentioned above it will in turn also require access to the Internet allowing students to not only access the system locally but also at home or remotely as mentioned in Human Access Channels.
	\end{itemize}
	\item External Plagiarism Checker:
	\begin{itemize}
		\item Developing a plagiarism checker from scratch is time consuming and also outside the scope of the system so in order to assess the posts for potential copyright infringement access to an external checker will need to be incorporated.
		\item TurnItIn is a company that provides such a service.
	\end{itemize}
\end{enumerate} 